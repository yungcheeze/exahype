\documentclass{scrreprt}

% usepackages
\usepackage{amsmath,amssymb}
\usepackage{subfiles}
\usepackage{nicefrac}
\usepackage[amsmath,framed,thmmarks]{ntheorem}  
\usepackage{hyperref}
\usepackage{framed}

% shortcuts

% math general
\newcommand{\partialup}{\partial}

% theorem environments
\theoremstyle{definition}
\theorembodyfont{\itshape}      % try commenting this line
\theoremprework{}       % code to process before the theorem 
\theorempostwork{}      % code to process after the theorem 
\theoremseparator{:}    % could be a : for example
\newtheorem{theorem}{Theorem}[chapter]
\newtheorem{problem}[theorem]{Problem}
\newtheorem{assumption}[theorem]{Assumption}
\newtheorem{remark}[theorem]{Remark}
\newtheorem{definition}[theorem]{Definition}
\newtheorem{lemma}[theorem]{Lemma}
\newtheorem{corollary}[theorem]{Corollary}

\newcommand{\qedhere}{\ifmmode\qed\else\hfill\proofSymbol\fi}
\theoremstyle{nonumberplain}
\theoremheaderfont{\itshape}
\theorembodyfont{\normalfont}
\theoremsymbol{\ensuremath{\square}}
\newtheorem{proof}{Proof}
\qedsymbol{\ensuremath{\square}}

% vectors, matrices
\newcommand{\scal}[1]{\mathit{#1}}
\renewcommand{\vec}[1]{{\textbf{#1}}}
\newcommand{\stvec}[1]{\widetilde{\vec{#1}}}
\newcommand{\gvec}[1]{{\boldsymbol #1}}
\newcommand{\mat}[1]{{\textbf #1}}
\newcommand{\matel}[3]{#1_{#2\,#3}}
\newcommand{\gmat}[1]{{\boldsymbol #1}}
\newcommand{\vecel}[2]{\mathit{#1}_{#2}}
\newcommand{\gmatel}[3]{#1_{#2\,#3}}
\newcommand{\transp}{^\textrm{T}}
\newcommand{\trace}{\text{tr}}
\newcommand{\nullvec}{\mathbf{0}}

% analysis
\newcommand{\suprem}[1]{\textup{sup}_{#1}}

% linear algebra
\newcommand{\laVec}[1]{\underline{\mathrm{#1}}}
\newcommand{\laVecel}[1]{\mathrm{#1}}
\newcommand{\laMat}[1]{\underline{\mathrm{#1}}}
\newcommand{\laMatel}[1]{\mathrm{#1}}
\newcommand{\laSpace}[1]{\mathbb{R}^{#1}}

% geometry
\newcommand{\point}{\vec{x}}
\newcommand{\domain}{\Omega}
\newcommand{\domainClosed}{\overline{\domain}}
\newcommand{\boundary}{{\partialup\domain}}
\newcommand{\boundaryD}{{\boundary_\textup{D}}}
\newcommand{\boundaryN}{{\boundary_\textup{N}}}
\newcommand{\boundaryM}{{\boundary_\textup{n}}}
\newcommand{\boundaryIn}{{\boundary_{-}}}
\newcommand{\boundaryOut}{{\boundary_{+}}}
\newcommand{\inDomain}{\textup{in }\domain}
\newcommand{\onBoundary}{\textup{on }\boundary}
\newcommand{\onBoundaryD}{\textup{on }\boundaryD}
\newcommand{\onBoundaryN}{\textup{on }\boundaryN}
\newcommand{\onBoundaryM}{\textup{on }\boundaryM}
\newcommand{\neumannVec}{\vec{g}_\textup{N}}
\newcommand{\neumannVecel}{g_{\textup{N},i}}
\newcommand{\dirichletVec}{\vec{g}_\textup{D}}
\newcommand{\dirichletVecel}{g_{\textup{D},i}}
\newcommand{\navierNeumann}{g_\textup{n,t}}
\newcommand{\navierDirichlet}{g_\textup{n,n}}
\newcommand{\inflow}{g_{-}}
\newcommand{\normal}{\vec{n}}
\newcommand{\normalT}[1]{\vecel{n}{#1}}
\newcommand{\tangential}{\vec{t}}
\newcommand{\tangentialT}[1]{\vecel{t}{#1}}

% mesh 
\newcommand{\triaClosed}{\bar{\mathcal{E}}_h}
\newcommand{\tria}{\mathcal{T}_h}
\newcommand{\cell}{K}
\newcommand{\Ncell}{{N_{\cell}}}
\newcommand{\cellClosed}{\bar{\cell}}
\newcommand{\cellNormal}[1]{\vec{n}^{{#1}}}
\newcommand{\cellNormalT}[2]{n_{#2}^{{#1}}}
\newcommand{\cellTangent}[1]{\vec{t}^{{#1}}}
\newcommand{\cellTangentT}[2]{t_{#2}^{{#1}}}
\newcommand{\cellPlus}{{\cell^+}}
\newcommand{\cellMinus}{{\cell^-}}
\newcommand{\cellBnd}{{\partialup\cell}}
\newcommand{\cellBndPlus}{{\partialup\cellPlus}}
\newcommand{\cellBndMinus}{{\partialup\cellMinus}}
\newcommand{\face}{e}
\newcommand{\nCellFaces}{{N_\face}}
\newcommand{\skeleton}{\Gamma}
\newcommand{\internalFaces}{\Gamma_h}
\newcommand{\skeletonD}{{\skeleton_\textup{D}}}
\newcommand{\skeletonN}{{\skeleton_\textup{N}}}
\newcommand{\skeletonM}{{\skeleton_\textup{n}}}
\newcommand{\sumDim}{\sum_{i=1}^d}
\newcommand{\sumDims}{\sum_{i,\idx{j}=1}^d}
\newcommand{\sumCellFaces}{\sum_{i=1}^{\nCellFaces}}
\newcommand{\sumCells}{\sum_{\cell \in \tria}}
\newcommand{\sumInternalFaces}{\sum_{\face\in\internalFaces}}
\newcommand{\sumSkeletonD}{\sum_{\face\in\skeletonD}}
\newcommand{\sumSkeletonM}{\sum_{\face\in\skeletonM}}
\newcommand{\sumSkeletonN}{\sum_{\face\in\skeletonN}}
\newcommand{\sumFaces}{\sum_{\face \in \internalFaces \cup \skeletonD}}
\newcommand{\sumFacesImpl}{\sum_{\face \in \internalFaces \cup \skeletonD \cup \skeletonM}}
\newcommand{\sumNeumann}{\sum_{\face \in \boundaryN}}
\newcommand{\hmax}{h_\textup{max}}
\newcommand{\hcell}{h_\cell}
\newcommand{\hcellPlus}{h_\cellPlus}
\newcommand{\hcellMinus}{h_\cellMinus}
\newcommand{\hface}{h_\face}
\newcommand{\kmax}{k_\textup{max}}
\newcommand{\kface}{k_\face}
\newcommand{\kcell}{k_\cell}
\newcommand{\kcellPlus}{k_\cellPlus}
\newcommand{\kcellMinus}{k_\cellMinus}
\newcommand{\lcell}{l_\cell}

\newcommand{\dt}{\textup{d}t}

% integrals
\newcommand{\dV}{\text{d}\vec{x}}
\newcommand{\ds}{\text{d}s}
\newcommand{\intDomain}{\int_\domain}
\newcommand{\intBoundary}{\int_\boundary}
\newcommand{\intCell}{\int_\cell}
\newcommand{\intFace}{\int_\face}

% mapping
\newcommand{\refVec}[1]{\hat{\vec{#1}}}
\newcommand{\refCell}{\hat{\cell}}
\newcommand{\refFace}{\hat{\face}}
\newcommand{\mapping}{\boldsymbol{\mathcal{F}}_{\cell}}
\newcommand{\mappingT}[1]{\mathcal{F}_{\cell,#1}}
\newcommand{\imapping}{\boldsymbol{\mathcal{F}}^{-1}_{\cell}}
\newcommand{\jacobian}{\textup{D}\boldsymbol{\mathcal{F}}_{\cell}}
\newcommand{\jacobianT}[2]{\frac{\partialup \mathcal{F}_{\cell,#1}}{\partialup
{#2}}}
\newcommand{\detJ}{J_\cell}
\newcommand{\ijacobian}{\textup{D}\vec{F}_{\cell}^{-1}}
\newcommand{\intRefCell}{\int_{\hat{\cell}}}
\newcommand{\refdV}{\textup{d}\hat{\vec{x}}}
\newcommand{\refdx}{\textup{d}\hat{x}}
\newcommand{\refdy}{\textup{d}\hat{y}}

\newcommand{\mappingst}{\boldsymbol{\mathcal{F}}_{\cell,t}}
\newcommand{\imappingst}{\boldsymbol{\mathcal{F}}^{-1}_{\cell,t}}

\newcommand{\reft}{\hat{t}}
\newcommand{\refdt}{\textup{d}\hat{t}}

% discrete function spaces
\newcommand{\spans}[1]{\textup{span}\left\{#1\right\}}
\newcommand{\polyspace}[3]{\mathit{#1}_{#2}(#3)}
\newcommand{\polyspaceVec}[3]{\vec{#1}_{#2}(#3)}

\begin{document}

\title{ADER-DG Implementation}
\author{Dominic Etienne Charrier}
\date{\today}
\maketitle

%%%%%%%%%%%%%%%%%%%%%%%%%%%%%%%%%%%%%%%%%%%%%%%%%%%%%%%%%%%%%%%
%%%%%%%%%%%%%%%%%%%%%%%%%%%%%%%%%%%%%%%%%%%%%%%%%%%%%%%%%%%%%%%
\chapter{ADER-DG Implementation}
%%%%%%%%%%%%%%%%%%%%%%%%%%%%%%%%%%%%%%%%%%%%%%%%%%%%%%%%%%%%%%%
%%%%%%%%%%%%%%%%%%%%%%%%%%%%%%%%%%%%%%%%%%%%%%%%%%%%%%%%%%%%%%%
\section{Notation}
\begin{itemize}
  \item Scalar quantities with physical meaning (coordinates, pressures,
  components of vectors with physical meaning,\ldots) are written in italic
  letters.
  This also includes the components of vectors and tensors with physical meaning;
  see the next bullet point.
  \item Vectors and tensors with physical meaning (points, velocities, forces,
  stresses,\ldots) are printed in bold and using upright letters, e.g.,
  $\mat{A}$, $\vec{f}$.
  \item Linear algebra vectors and matrices are written in upright letters
  and are underlined, e.g., $\laMat{A}$, $\laVec{f}$. Elements belonging to
  linear algebra vectors and matrices
  are also written in an upright font but are not underlined, e.g.
  $\laMatel{A}_{ij}$, $\laVecel{f}_i$.
\end{itemize}
%%%%%%%%%%%%%%%%%%%%%%%%%%%%%%%%%%%%%%%%%%%%%%%%%%%%%%%%%%%
%%%%%%%%%%%%%%%%%%%%%%%%%%%%%%%%%%%%%%%%%%%%%%%%%%%%%%%%%%%
\section{Computational mesh and trace operators}
\label{sec:mesh}
%%%%%%%%%%%%%%%%%%%%%%%%%%%%%%%%%%%%%%%%%%%%%%%%%%%%%%%%%%%
%%%%%%%%%%%%%%%%%%%%%%%%%%%%%%%%%%%%%%%%%%%%%%%%%%%%%%%%%%%
Let $\tria$ be a partition of the domain
$\domain\subset\mathbb{R}^d$ consisting of quadrilateral/hexahedral elements.
We refer to the disjoint open sets $\cell \in \tria$ as grid cells. We denote by
$\hcell$ the diameter of $\cell$.
We store the local quantities $\hcell$ in the vector
$\laVec{h} = \{ \hcell \}_{\cell \in \tria}$, and
set
$h_\textup{max} = \text{max}_{\cell \in \tria }\,\hcell$.
Finally, $\normal$ denotes the outward normal unit vector to the cell boundary
$\cellBnd$. We will denote the number of cells by $\Ncell$.

An interior face of $\tria$ is the $d-1$ dimensional intersection
$\cellBnd^{+} \cap \cellBnd^{-}$, where $\cellPlus$ and $\cellMinus$
are two adjacent elements of $\tria$. Similarly, a boundary face of $\tria$ is
the $d-1$ dimensional intersection $\cellBnd \cap \boundary$ which consists
of entire faces of $\cellBnd$.
We denote by $\internalFaces$ the union of all interior faces of $\tria$.

Here and in the following, we refer generically to a ``face" even in the
case $d=2$.
%%%%%%%%%%%%%%%%%%%%%%%%%%%%%%%%%%%%%%%%%%%%%%%%%%%%%%%%%%%
%%%%%%%%%%%%%%%%%%%%%%%%%%%%%%%%%%%%%%%%%%%%%%%%%%%%%%%%%%%
\section{Mappings}
%%%%%%%%%%%%%%%%%%%%%%%%%%%%%%%%%%%%%%%%%%%%%%%%%%%%%%%%%%%
%%%%%%%%%%%%%%%%%%%%%%%%%%%%%%%%%%%%%%%%%%%%%%%%%%%%%%%%%%%
In this section, we introduce mappings between the grid cells
$\cell\in\tria$ and a reference cell. Mappings allow
us to treat integrals over grid cells with non-uniform extent in an
uniform manner.
To ease the presentation, we only consider the two-dimensional case here.
\\[5pt]
Let us define the \textit{reference cell} ${\refCell=[0,1]^2}$
and let us denote by $\refVec{x}=(\hat{x},\hat{y})\transp$ a point belonging to $\refCell$.
Let $\cell\in\tria$ denote a nondegenerate quadrilateral cell with
center $\vec{P}_0$ and cell size $(\Delta x,\,\Delta y)^T$.

We can express $\cell$ according to
\begin{align}
\cell &= \mapping (\refCell),
\notag
\intertext{where we define the mapping $\mapping : \refCell \to \cell$ as}
\label{eq:aderdg_impl:mapping}
\vec{x} = \mapping(\refVec{x})
&=
\begin{pmatrix}
\mappingT{x}(\refVec{x})\\
\mappingT{y}(\refVec{x})
\end{pmatrix}
=
\vec{P}_0 +
\begin{pmatrix}
\Delta x & 0 \\
0 & \Delta y
\end{pmatrix}
\begin{pmatrix}
\hat{x}-0.5 \\ \hat{y}-0.5
\end{pmatrix}
\end{align}

We denote the inverse mapping by $\imapping\colon\,\cell \to \refCell$.
It is easy to see that the considered elements have a constant Jacobian matrix,
\begin{align}
\jacobian (\refVec{x})&=
\begin{pmatrix}
\jacobianT{x}{\hat{x}} & \jacobianT{x}{\hat{y}} \\
\jacobianT{y}{\hat{x}} & \jacobianT{y}{\hat{y}}
\end{pmatrix}(\refVec{x})
=
\begin{pmatrix}
\Delta x & 0 \\
0 & \Delta y
\end{pmatrix},
\notag
\intertext{and thus also a constant Jacobian determinant}
\detJ(\refVec{x}) = \textup{det}\left(\jacobian\right)(\refVec{x})&=
\Delta x\,\Delta y.
\notag
\end{align}

For the affine cells considered here, we can easily proof the
following identity.
Let $\hat{f}$ be a sufficiently regular function on the reference cell $\refCell$, and
let $f$ be sufficiently regular on $\cell$ and such that
\begin{align}
\label{eq:aderdg_impl:imapping_scalar_function}
f(\vec{x}) &= (\hat{f} \circ \imapping)(\vec{x}) = \hat{f}(\refVec{x}),
\qquad
\refVec{x}\in\refCell,\,\vec{x}=\mapping(\refVec{x})\in\cell.
\end{align}
Then, it holds that
\begin{align}
\nabla\,f(\vec{x})
&= \left(\jacobian^{-\textup{T}}\,\cdot\hat{\nabla}\,\hat{f}\right)(\refVec{x})
=
\begin{pmatrix}
\frac{1}{\Delta x} & 0 \\
0 & \frac{1}{\Delta y}
\end{pmatrix}\,
\hat{\nabla}\,\hat{f}(\refVec{x})
\notag
\intertext{where}
\hat{\nabla} &= \left(\frac{\partialup}{\partialup \hat{x}},\,\frac{\partialup}{\partialup \hat{y}}\right)\transp
\notag
\end{align}
denotes the gradient with respect to the reference coordinates. This follows
from \eqref{eq:aderdg_impl:imapping_scalar_function} and using the chain rule.
A similar identity can be derived for vector-valued functions.
We can further express volume integrals over the element $\cell$
with respect to the reference coordinates and the invertible mapping $\mapping$.
Let $\hat{f}$ and $f$ be defined as in \eqref{eq:aderdg_impl:imapping_scalar_function},
we have
\begin{align}
\intCell f(\vec{x})\,\dV &= \intRefCell \hat{f}(\refVec{x})\,|\detJ(\refVec{x})|\,\refdV
=
\int_{0}^1\int_{0}^1 \hat{f}(\hat{x},\hat{y})\,|\detJ(\hat{x},\hat{y})|\,\refdV\,\refdy.
\notag
\end{align}
Further, if we define the faces of the reference cell according to
\begin{align}
\refFace_{1} &= \{ \refVec{x} \in \refCell : \hat{x} = 0\},\qquad
\refFace_{3} =  \{ \refVec{x} \in \refCell : \hat{y} = 0\},
\notag\\
\refFace_{2} &= \{ \refVec{x} \in \refCell : \hat{x} = 1\},\qquad
\refFace_{4} =  \{ \refVec{x} \in \refCell : \hat{y} = 1\},
\notag
\end{align}
we can define the four faces belonging to the boundary of $\cell$ by
\begin{align}
\face_{1} &= \mapping(\refFace_1),\qquad
\face_{2} = \mapping(\refFace_2),\qquad
\face_{3} = \mapping(\refFace_3),\qquad
\face_{4} = \mapping(\refFace_4).
\notag
\end{align}
Now we can express integrals over faces belonging to the boundary of $\cell$ with respect to
the reference coordinates and the mapping $\mapping$.
Let $\hat{f}$ and $f$ be defined as in \eqref{eq:aderdg_impl:imapping_scalar_function},
we have
\begin{align}
\int_{\face_1} f(\vec{x})\,\ds &=
\int_{0}^1\hat{f}(0,\hat{x})\,|\detJ(0,\hat{x})|\,\refdx,
\notag\\
\int_{\face_2} f(\vec{x})\,\ds &=
\int_{0}^1\hat{f}(1,\hat{x})\,|\detJ(1,\hat{x})|\,\refdx,
\notag\\
\int_{\face_3} f(\vec{x})\,\ds &=
\int_{0}^1\hat{f}(\hat{y},0)\,|\detJ(\hat{y},0)|\,\refdy,
\notag\\
\int_{\face_4} f(\vec{x})\,\ds &=
\int_{0}^1\hat{f}(\hat{y},1)\,|\detJ(\hat{y},1)|\,\refdy.
\notag
\end{align}
Please note that all concepts considered in this chapter naturally extend to the
three-dimensional case.

\subsection{Space-time mappings}
We can express the tuple $(\cell,t)$ according to
\begin{align}
(\cell,t) &= \mappingst (\refCell,\hat{t}),
\notag
\intertext{where we define the mapping $\mappingst\colon\, \refCell\times[0,1]
\to \cell\times[t^K,t^K+\Delta t]$ according to}
\label{eq:aderdg_impl:mapping}
(\vec{x},t) = \mappingst(\refVec{x})
&=
(\vec{P}_0, t^K) +
\begin{pmatrix}
\Delta x &        0 &        0 \\
0        & \Delta y &        0  \\
0        &        0 & \Delta t
\end{pmatrix}
\begin{pmatrix}
\hat{x}-0.5 \\ \hat{y}-0.5 \\ \hat{t}
\end{pmatrix}.
\end{align}

\section{Basis functions}
Let be $f$ be a sufficiently regular univariate function such that it
can be approximated by a polynomial of order $N$, i.e.,
the leading approximation error term is of order $\mathcal{O}(N+1)$.

Given a set of support point and function value pairs $\{(x_i,f(x_i))\}_{0\leq
i\leq N}$, the corresponding interpolation polynomial in the Lagrange form
can be constructed according to:
\begin{align}
f_N (x) &= \sum_{i=0}^{N}\,f(x_i)\,\varphi_i(x),
\intertext{where the Lagrange basis polynomials are defined by}
L_i(x) &=
\left(\prod_{\begin{smallmatrix}0\le l\le N\\ l\neq i\end{smallmatrix}}
\frac{x-x_l}{x_i-x_l}\right),
\qquad i = 0,\ldots,N,
\notag
\intertext{Since we exclude the $(x-x_i)$ term in the product, the basis
functions have the property} L_i(x_j) &=
\label{eq:aderdg_impl:lagrange_basis:discrete_nodal_orthogonality}
\delta_{ij}
=
\begin{cases}
1 &i=j,\\
0 &\textup{else}.
\end{cases}
\qquad i,j = 0,\ldots,N,
\end{align}

Let $I\in\mathbb{R}$ be a open or closed real interval.
With respect to the Gauss-Legendre quadrature of degree $2\,(N+1)-1$,
we define a discrete scalar product
$\langle \cdot,\cdot \rangle_{L^2(I)}$ according to:
\begin{align*}
\langle f,g\rangle_{L^2(K)} =
\sum_{n=0}^N\,w_n\,f(x_n)\,g(x_n),
\end{align*}
where $x_n$ denotes a quadrature node and $w_n$ denotes a quadrature
weight, $n=0,1,\ldots,N$.

The next result is very important for the definition of the DG basis
functions:
\begin{lemma}
Let us denote by $\{\varphi_{i}\}_{i=0,1,\ldots,N}$ a Lagrange basis utilising
basis polynomials located at the nodes of a Gauss-Legendre quadrature of
degree $2\,(N+1)-1$. Let the nodes lie in the interval $I$.
Then, $\{\varphi_{i}\}_{i=0,1,\ldots,N}$ is a orthogonal
basis with respect to the $(\cdot,\cdot)_{L^2(I)}$ scalar product.
Furthermore, it is a orthogonal basis with respect to all
Gauss-Legendre quadratures with at least a degree
of $2\,(N+1)-1$ that are used to evaluate $(\cdot,\cdot)_{L^2(I)}$.
\end{lemma}
\begin{proof}
We only need to proof that the basis functions are orthogonal with
respect to the Gauss-Legendre quadrature of degree $2\,(N+1)-1$.
The rest follows from the fact that the Gauss-Legendre
quadrature of degree ${2\,(N+1)-1}$ is exact for all polynomials of degree
$2\,(N+1)-1 > 2\,N$, where $2\,N$ is the maximum total degree
of the product of two basis polynomials.

Select two basis functions $\varphi_{i}$ and
$\varphi_{j}$,$i,j=0,1,\ldots,N$.
Since both functions are Lagrange polynomials, we have
\begin{align}
\label{eq:preliminaries:lagrange_orthogonality}
\varphi_i (x_j) =
\delta_{ij} =
\begin{cases}
1 & i=j, \\
0 & \textup{else},
\end{cases}
\end{align}
where $x_j$ denotes a support point, and $\delta_{ij}$ denotes the
Kronecker delta.
An analogous condition holds for $\varphi_j$.

We want to evaluate the scalar product
$(\cdot,\cdot)_{L^2(K)}$ by using the Gauss-Legendre
quadrature of degree $2\,(N+1)-1$.
We obtain:
\begin{align*}
(\varphi_i,\,\varphi_j)_{L^2(K)}
&=
\langle\varphi_i,\,\varphi_j\rangle_{L^2(K)}
=
\sum_{n=0}^{N} w_n\,\varphi_{i}(x_n)\,\varphi_{j}(x_n) =
\sum_{n=0}^{N} w_n\,\delta_{ij}\,\delta_{nj} =
w_i\,\delta_{ij}.
\end{align*}
\end{proof}
\subsection{Definition of the DG basis functions}
Let us introduce the space of polynomials of order
at most $N$ with support in $[0,\,1]$:
\begin{align*}
\polyspace{Q}{N}{[0,\,1]} &= \text{span}\left\{
\hat{x}^{n}\; \text{ such that }
0\leq n \leq N \textup{ and }\hat{x} \in [0,\,1] \right\}.
%\polyspace{Q}{N}{\refCell} &= \text{span}\left\{ \Pi_{i=1}^d\,
%x_i^{n_i}\; \text{ such that }n_i \in \mathbb{N}_0 \text{ and }
%n_i \leq N,\,i=1,\ldots,d,\, \text{ and } \vec{x} \in \cell \right\}.
\end{align*}
As a basis of the space
$\polyspace{Q}{N}{[0,1]}$,
we use a set of $N+1$ Lagrange basis polynomials.
The support points of the basis polynomials are chosen such that
they coincide with the nodes of a Gauss-Legendre quadrature of degree
$2\,(N+1)-1$ which uses the interval $[0,\,1]$
as domain of integration. We denote the scalar-valued univariate reference basis
functions by $\hat{\varphi}_i(\hat{x})$, $i=0,1,\ldots,N$.

Let us define scalar-valued multivariate polynomials on the reference cell
$\refCell=[0,1]^d$,
and on each grid cell $\cell\in\tria$
according to:
\begin{align}
\label{eq:ader_impl:basis:basis_definition_ref_3d}
\hat{\phi}_n (\refVec{x}) &=
\prod^{d}_{\xi=1}
\hat{\varphi}_{n_\xi} (\hat{x}_\xi),
\\
\label{eq:ader_impl:basis:basis_definition_grid_3d}
\phi^{K}_n (\vec{x}) &=
\begin{cases}
(\hat{\phi}_n \circ \imapping) (\vec{x}) & \vec{x}\in\cell,
\\
0 & \textup{else},
\end{cases}
\end{align}
for $n_\xi=0,1,\ldots,N$, $n_\xi\in\{1,\ldots,d\}$,
and a linearised index $n=0,1,\ldots,(N+1)^{d}-1$
that is constructed according to:
\begin{align*}
n = \sum_{\xi=1}^{d} M_\xi\,n_\xi,
\end{align*}
using the one-dimensional basis indices $n_\xi={0,1,\ldots,N}$ and some strides
$M_\xi$, $\xi \in \{1,2,\ldots,d\}$ that define an unique order of the degrees
of freedom.

Notice that \eqref{eq:ader_impl:basis:basis_definition_grid_3d} implies that
\begin{align*}
\vec{x} = \mapping \refVec{x}
\qquad
\Leftrightarrow
\qquad
\phi^{K}_n (\vec{x}) =
\hat{\phi}_n (\refVec{x}),
\end{align*}
for $\vec{x}\in\cell$, $\refVec{x}\in\refCell$,
and $n=0,1,\ldots,(N+1)^{d}-1$.

The corresponding $d$-dimensional support points ($d$-dimensional
Gauss-Legendre quadrature nodes) on the reference cell are constructed as tensor
products of the one-dimensional support points:
\begin{align*}
\refVec{x}_n = (\hat{x}_{n,1},\ldots,\hat{x}_{n,d})\transp,
\end{align*}
with $n=0,1,\ldots,(N+1)^{d}-1$.

One can show that the polynomials $\{\hat{\phi}_n
(\refVec{x})\}_{n=0,1,\ldots,(N+1)^d-1}$ form a basis of the space
\begin{align*}
\polyspace{Q}{N}{\refCell} &= \text{span}\left\{ \Pi_{i=1}^d\,
\hat{x}_i^{n_i}\; \text{ such that }
0 \leq n_i \leq N,\,i=1,\ldots,d,\, \text{ and } \refVec{x} \in \refCell
\right\}
\end{align*}
The corresponding $d$-dimensional support points ($d$-dimensional
Gauss-Legendre quadrature nodes) on the reference cell are constructed as tensor
products of the one-dimensional quadrature nodes:
\begin{align*}
\refVec{x}_n = (\hat{x}_{n,1},\ldots,\hat{x}_{n,d})\transp,
%\intertext{and the corresponding quadrature weights are constructed according
%to:} w_n = \prod_{\xi=1}^{d} w_{n_\xi},
\end{align*}
with $n=0,1\ldots,(N+1)^d-1$ denoting the global index.

Since we only consider affine reference cell to grid cell mappings, we can
further show that the polynomials $\{{\phi}^K_n
(\vec{x})\}_{n=0,1,\ldots,(N+1)^d-1}$ form a basis of the space
\begin{align*}
\polyspace{Q}{N}{\cell} &= \text{span}\left\{ \Pi_{i=1}^d\,
x_i^{n_i}\; \text{ such that }
0 \leq n_i \leq N,\,i=1,\ldots,d,\, \text{ and } \vec{x} \in \cell
\right\}.
\end{align*}
Let us introduce another index $v=0,1,\ldots,N_\textup{var}-1$
that numbers the variables.
We construct the basis $\{{\phi}^{K;v}_n
(\vec{x})\}_{v=0,1,\ldots,N_\textup{var}-1;\;n=0,1,\ldots,(N+1)^d-1}$
of the space
$\polyspace{Q}{N}{\cell}^{N_\textup{var}}$ according to:
\begin{align*}
\{{\phi}^{K;v}_n
(\vec{x})\}&_{v=0,1,\ldots,N_\textup{var}-1;\;n=0,1,\ldots,(N+1)^d-1}
= \\
&\lbrace
(
1,\,
0,\,
\ldots,
0
)_{N_\textup{var}}\transp
,
(
0,\,
1,\,
\ldots,
0
)_{N_\textup{var}}\transp
,
\ldots,
,
(
0,\,
0,\,
\ldots,
1
)_{N_\textup{var}}\transp
\rbrace
\,\otimes
\\
&\{{\phi}^K_n
(\vec{x})\}_{n=0,1,\ldots,(N+1)^d-1}
\end{align*}

We further construct a basis $\{{\phi}^{K;v;e}_n
(\vec{x})\}_{e=1,2,\ldots,d;\;v=0,1,\ldots,N_\textup{var}-1;\;n=0,1,\ldots,(N+1)^d-1}$
of the space
$\polyspace{Q}{N}{\cell}^{N_\textup{var}\times d}$ according to:
\begin{align*}
\{{\phi}^{K;v;e}_n
(\vec{x})\}&_{e=1,2,\ldots,d;\;v=0,1,\ldots,N_\textup{var}-1;\;n=0,1,\ldots,(N+1)^d-1}
=
\\
&\lbrace
(
1,\,
0,\,
\ldots,
0
)_{N_\textup{var}}\transp
,
(
0,\,
1,\,
\ldots,
0
)_{N_\textup{var}}\transp
,
\ldots,
,
(
0,\,
0,\,
\ldots,
1
)_{N_\textup{var}}\transp
\rbrace
\,\otimes
\\
&\lbrace
(
1,\,
0,\,
\ldots,
0
)_{d}\transp
,
(
0,\,
1,\,
\ldots,
0
)_{d}\transp
,
\ldots
,
(
0,\,
0,\,
\ldots,
1
)_{d}\transp
\rbrace
\,\otimes
\\
&\{{\phi}^{K}_n(\vec{x})\}_{n=0,1,\ldots,(N+1)^d-1}
\end{align*}

Lastly, let us introduce the space-time basis polynomials
\begin{align*}
\hat{\theta}_{l;n} (\refVec{x},\hat{t}) &=
\hat{\varphi}_{l} (\hat{t})
\,
\hat{\phi}_{l;n} (\refVec{x},\hat{t})
\\
\theta^{K}_{l;n} (\vec{x},t) &=
\begin{cases}
(\hat{\theta}_{l;n} \circ \imappingst) (\vec{x},t) &
(\vec{x},t)\in\cell \times [t^K,t^K+\Delta t]  \\
0 & \textup{else},
\end{cases}
\end{align*}
with $l=0,1,\ldots,N$, and $n=0,1,\ldots,(N+1)^d-1$.
Here, $t^K$ denotes a time stamp associated with
the cell $\cell$.

The corresponding $(d+1)$-dimensional support points ($(d+1)$-dimensional
Gauss-Legendre quadrature nodes) on the reference cell are constructed as tensor
products of the one-dimensional support points:
\begin{align*}
(\refVec{x}_{n},\hat{t}_l) =
(\hat{x}_{n,1},\ldots,\hat{x}_{n,d},\hat{t}_l)\transp,
\end{align*}
with $n\in\{0,1,\ldots,(N+1)^{d}-1\}$ denoting the index of the spatial basis
and with $l\in\{0,1,\ldots,N\}$ denoting the index of the temporal basis.

We can now construct a basis $\{{\theta}^{K;v}_{l;n}
(\vec{x})\}_{v=0,1,\ldots,N_\textup{var}-1;\;l=0,1,\ldots,N;\;n=0,1,\ldots,(N+1)^d-1}$
of the space
$\polyspace{Q}{N}{\cell\times[t^K,t^K+\Delta t]}^{N_\textup{var}}$ according to:
\begin{align*}
\{{\theta}^{K;v}_{l;n}
(\vec{x})\}&_{v=0,1,\ldots,N_\textup{var}-1;\;l=0,1,\ldots,N;\;n=0,1,\ldots,(N+1)^d-1}
= \\
&\lbrace
(
1,\,
0,\,
\ldots,
0
)_{N_\textup{var}}\transp
,
(
0,\,
1,\,
\ldots,
0
)_{N_\textup{var}}\transp
,
\ldots,
,
(
0,\,
0,\,
\ldots,
1
)_{N_\textup{var}}\transp
\rbrace
\\
&\otimes
\{{\theta}^K_{l;n}
(\vec{x})\}_{l=0,1,\ldots,N;\;n=0,1,\ldots,(N+1)^d-1}.
\end{align*}
%%%%%%%%%%%%%%%%%%%%%%%%%%%%%%%%%%%%%%%%%%%%%%%%%%%%%%%%%%%
\subsection{Properties of the basis functions}
%%%%%%%%%%%%%%%%%%%%%%%%%%%%%%%%%%%%%%%%%%%%%%%%%%%%%%%%%%%
Let us number the $d$-dimensional Gauss-Legendre quadrature weights on the
reference cell with the linearised index
$n=0,1,\ldots,(N+1)^{d}-1$ we used to number the basis functions
and the corresponding $d$-dimensional spatial support points, i.e.,
\begin{align}
w_n = \prod_{\xi=1}w_{n_\xi},
\end{align}
with $n_\xi=0,1,\ldots,N$, $n_\xi\in\{1,\ldots,d\}$.

In this section, we summarise the properties of most of the
basis polynomials that appeared in the previous section.
We exclude the basis polynomials of the volume flux
and space-time volume flux ansatz spaces in the summary.
In the following, let $l,l'\in\{0,1,\ldots,N\}$,
$n,n'\in\{0,1,\ldots,(N+1)^d-1\}$,
$\cell,\cell'\in\tria$ and $v,v'\in\{0,1,\ldots,N_\textup{var}\}$.
The basis functions have the following properties:
\begin{itemize}
  \item Lagrange basis property (reference cell):
  \begin{align}
\label{eq:ader_impl:basis:lagrange_ref_1d}
{\hat{\varphi}}_{l} (\hat{x}_{l'}) &= \delta_{ll'},\\
{\hat{\phi}}_{n} (\refVec{x}_{n'}) &= \prod_{\xi=1}^d
{\hat{\varphi}}_{n_{\xi}} (\hat{x}_{m,\xi})
=
\prod_{\xi=1}^{d}
\delta_{n_\xi n'_\xi}
= \delta_{n n'},
\\
{\hat{\theta}}_{l;n} (\refVec{x}_{n'},\hat{t}_{l'}) &=
\hat{\varphi}_{l}(\hat{t}_{l'})\,{\hat{\phi}}_{n'} (\refVec{x}_{n'})
= \delta_{l l'}\,\delta_{n n'}.
\end{align}
\item Sampling property (reference cell):
\begin{align}
\label{eq:ader_impl:basis:sampling_ref_1d}
\langle \hat{f},\,\hat{\varphi}_{l'}\rangle_{L^2([0,1])}   &=
w_{l'}\,\hat{f}(\hat{x}_{l'}),
\\
%%%%%%%%
\langle \hat{f},\,\hat{\phi}_{n'}\rangle_{L^2(\refCell)}
&=
\prod_{\xi=1}^d
\langle \hat{f},\,\hat{\varphi}_{n'_\xi}\rangle_{L^2([0,1])}
\\
&=
w_{n}\,
\hat{f}(\refVec{x}_n),
\\
%%%%%%%%
\langle\hat{f},\,\hat{\theta}_{l';n'}\rangle
_{L^2(\refCell\times[0,1])}
&= w_{l}\,w_{n}\,
\hat{f}(\refVec{x}_n,t_l),
\end{align}
\item Sampling property (grid cell):
\begin{align}
\label{eq:ader_impl:basis:sampling_grid_1d}
\langle f,\,\varphi_{l}\rangle_{L^2([t_0,t_0+\Delta t])}   &=
\Delta t\,w_l\,f(t_l),
\\
%%%%%%%%
\label{eq:ader_impl:basis:sampling_grid_3d}
\langle f,\,\phi_{n}\rangle_{L^2(\cell)}
&=
\detJ\,
w_{n}\,
f(\vec{x}_n),
\\
%%%%%%%%
\langle f,\,\theta_{l';n'}\rangle
_{L^2(\cell\times[t^\cell,t^\cell+\Delta t])}
&=
\Delta t\,\detJ\,
w_{l'}\,w_{n'}\,
f(\vec{x}_{n'},t_{l'}),
\end{align}
\item Discrete orthogonality (reference cell)
\begin{align}
\label{eq:ader_impl:basis:discrete_ortho_ref_1d}
\langle \hat{\varphi}_{l},\,\hat{\varphi}_{l'}\rangle_{L^2([0,1])}   &=
w_l\,\delta_{ll'},
\\
%%%%%%%%
\label{eq:ader_impl:basis:discrete_ortho_ref_3d}
\langle \hat{\phi}_{n},\,\hat{\phi}_{n'}\rangle_{L^2(\refCell)}
&= \prod_{\xi=1}^d
\langle \hat{\varphi}_{n_\xi},\,\hat{\varphi}_{n'_\xi}\rangle_{L^2([0,1])}
=
\prod_{\xi=1}^{d}
w_{n_\xi}
\delta_{n_\xi n'_\xi}\notag
\\
&=
w_{n}
\,\delta_{n n'}
,
%%%%%%%%
\\
\langle\hat{\theta}_{l;n},\,\hat{\theta}_{l';n'}\rangle
_{L^2(\refCell\times[0,1])}
&=
\langle \hat{\varphi}_{l},\,\hat{\varphi}_{l'}\rangle_{L^2([0,1])}
\,
\langle \hat{\phi}_{n},\,\hat{\phi}_{n'}\rangle_{L^2(\refCell)}\notag
\\
&= w_{l}\,w_{n}\,
\delta_{l l'}\,\delta_{n n'}.
\end{align}
%%%%%%%%%%%%%%%%
\item Discrete orthogonality (grid cell)
\begin{align}
\langle \varphi^\cell_{l},\,\varphi^\cell_{l'}\rangle_{L^2([t^K,t^K+\Delta
t])} &= \Delta t\,w_l\,\delta_{ll'},
\\
%%%%%%%%
\label{eq:ader_impl:basis:discrete_ortho_grid_3d}
\langle \phi^\cell_{n},\,\phi^\cell_{n'}\rangle_{L^2(\cell)}
&=
\detJ\,\langle \hat{\phi}_{n},\,\hat{\phi}_{n'}\rangle_{L^2(\refCell)}\notag
\\
&=
\detJ\,
w_{n}\,
\delta_{n n'},
\\
%%%%%%%%
\langle\theta^{K;v}_{l;n},\,\theta^{K;v}_{l';n'}\rangle
_{L^2(\cell\times[t^K,t^K+\Delta t])}
&=
\Delta t\,
\langle \hat{\varphi}_{l},\,\hat{\varphi}_{l'}\rangle_{L^2([0,\,1])}
\,
\detJ\,
\langle \hat{\phi}_{n},\,\hat{\phi}_{n'}\rangle_{L^2(\refCell)}\notag
\\
&= \Delta t\,\detJ\,w_{l}\,w_{n}\,\delta_{l l'}\,\delta_{n n'}.
\end{align}
\item Compact support:
\begin{align}
\langle \phi^{\cell}_{n},\,\phi^{\cell'}_{n'}\rangle_{L^2(\cell)}
&=
\langle \phi^{\cell}_{n},\,\phi^{\cell}_{n'}\rangle_{L^2(\cell)}
\,\delta_{\cell\cell'}
\notag
\\
&=
\detJ\,
w_{n}\,
\delta_{\cell\cell'}\,
\delta_{n n'},
\\
%%%%%%%%
\langle\theta^{\cell}_{l;n},\,\theta^{\cell'}_{l';n'}\rangle
_{L^2(\cell\times[t^\cell,t^\cell+\Delta t])}
&=
\Delta t\,
\langle \hat{\varphi}_{l},\,\hat{\varphi}_{l'}\rangle_{L^2([0,\,1])}
\,
\detJ\,
\langle \hat{\phi}_{n},\,\hat{\phi}_{n'}\rangle_{L^2(\refCell)}
\delta_{\cell\cell'}
\notag
\\
&=
\Delta t\,\detJ\,w_{l}\,w_{n}\,
\delta_{\cell\cell'}\,\delta_{l l'}\,\delta_{n n'}.
\end{align}
\item Orthogonal variables:
\begin{align}
\label{eq:ader_impl:basis:orthogonal_variables}
\langle \phi^{\cell;v}_{n},\,\phi^{\cell';v'}_{n'}\rangle_{L^2(\cell)}
&=
\delta_{vv'}\,\langle \phi^{\cell}_{n},\,\phi^{\cell}_{n'}\rangle_{L^2(\cell)}
\notag
\\
&=
\detJ\,
w_{n}\,
\delta_{\cell\cell'}\,
\delta_{vv'}\,
\delta_{n n'},
\\
%%%%%%%%%%%%%
\label{eq:ader_impl:basis:orthogonal_variables_spacetime}
\langle\theta^{\cell;v}_{l;n},\,\theta^{\cell';v'}_{l';n'}\rangle
_{L^2(\cell\times[t^\cell,t^\cell+\Delta t])}
&=
\langle\theta^{\cell}_{l;n},\,\theta^{\cell'}_{l';n'}\rangle
_{L^2(\cell\times[t^\cell,t^\cell+\Delta t])}\,
\delta_{vv'}
\notag
\\
&= \Delta t\,\detJ\,w_{l}\,w_{n}\,
\delta_{\cell\cell'}\,\delta_{vv'}\,\delta_{l l'}\,\delta_{n n'}.
\end{align}
\end{itemize}
%%%%%%%%%%%%%%%%%%%%%%%%%%%%%%%%%%%%%%%%%%%%%%%%%%%%%%%%%%%
\section{Operators}
%%%%%%%%%%%%%%%%%%%%%%%%%%%%%%%%%%%%%%%%%%%%%%%%%%%%%%%%%%%
In this section, we introduce operators that appear
within the derivation of the ADER-DG
operators and vectors.

Below, let $l,l'\in\{0,1,\ldots,N\}$,
$n,n'\in\{0,1,\ldots,(N+1)^d-1\}$,
$\cell,\cell'\in\tria$ and $v,v'\in\{0,1,\ldots,N_\textup{var}\}$.

First, let us introduce an operator $\hat{P}\in\mathbb{R}^{(N+1)\times(N+1)}$
that projects the coefficients associated
with the univariate reference basis functions on 
the $(N+1)$ nodes of a regular partition of interval $[0,1]$:
\begin{align}
\label{eq:ader_impl:operators:regular_grid_projector_1d}
\hat{P}_{ij} &= \varphi_i \left({\frac{j}{N}}\right),\qquad 
i,j=0,1,\ldots,N.
\end{align}

Let us further introduce the operators
\begin{align}
\left[\hat{f},\hat{g}\right]^{\tau}_{L^2(\refCell)}
&=
\int_{\refCell}
\hat{f}(\refVec{x},\tau)\,\hat{g}(\refVec{x},\tau)\,\refdV,
\\
\left[f,g\right]^{\tau}_{L^2(\cell)}
&=
\int_{\cell}
f(\vec{x},t^K+\Delta t\,\tau)\,g(\vec{x},t^K+\Delta t\,\tau)\,\dV,
\end{align}
where $\tau\in\{0,\,1\}$, $\hat{f}$ and $\hat{g}$ are square integrable on
$\refCell\times[0,\,1]$, and
$f$ and $g$ are square integrable on
$\cell\times[t^\cell,\,t^\cell+\Delta t]$.

For our basis polynomials, we obtain
\begin{align}
\label{eq:ader_impl:operators:tau_operator_1_ref}
\left[\hat{\theta}_{l;n},\,\hat{\phi}_{n'}\right]^{\tau}_{L^2(\refCell)}
&=
\hat{\varphi}_{l}(\tau)\,
\,
\langle
\hat{\phi}_{n},\,
\hat{\phi}_{n'}\,
\rangle_{L^2(\refCell)}
\notag
\\
&=
\hat{\varphi}_{l}(\tau)\,
\langle
\hat{\phi}_{n},\,
\hat{\phi}_{n'}\,
\rangle_{L^2(\refCell)}
\notag
\\
&=
w_{n}\,
\hat{F}^\tau_l\,
\delta_{n n'},
\\
%%%%%%
\label{eq:ader_impl:operators:tau_operator_1_grid}
\left[\theta^{\cell;v}_{l;n},\,\phi^{\cell;v}_{n'}\right]^{\tau}_{L^2(\cell)}
&=
\detJ\,
\left[\hat{\theta}_{l;n},\,\hat{\phi}_{n'}\left]^{\tau}_{L^2(\refCell)}
\notag
\\
&=
\detJ\,
w_{n}\,
\hat{F}^\tau_l\,
\delta_{\cell\cell'}\,\delta_{vv'}\,
\delta_{n n'},
\\
%%%%%%
\label{eq:ader_impl:operators:tau_operator_2_ref}
\left[\hat{\theta}_{l;n},\,\hat{\theta}_{l';n'}\right]^{\tau}_{L^2(\refCell)}
&=
\hat{\varphi}_{l}(\tau)\,
\hat{\varphi}_{l'}(\tau)\,
\,
\langle
\hat{\phi}_{n},\,
\hat{\phi}_{n'}\,
\rangle_{L^2(\refCell)}
\notag
\\
&=
w_{n}\,
\hat{F}^\tau_{l}\,
\hat{F}^\tau_{l'}\,
\delta_{n n'},
\\
%%%%%%
\label{eq:ader_impl:operators:tau_operator_2_grid}
\left[\theta^{\cell;v}_{l;n},\,\theta^{\cell;v}_{l';n'}\right]^{\tau}_{L^2(\cell)}
&=
\detJ
\,
\left[\hat{\theta}_{l;n},\,\hat{\theta}_{l';n'}\right]^{\tau}_{L^2(\refCell)}
\notag
\\
&=
\detJ\,
w_{n}\,
\hat{F}^\tau_{l}\,
\hat{F}^\tau_{l'}\,
\delta_{\cell\cell'}\,\delta_{vv'}\,
\delta_{n n'},
\end{align}
where we have introduced the vectors $\hat{F}^f \in \mathbb{R}^{N+1}$,
$f=0,1$, with
\begin{align}
\label{eq:ader_impl:operators:boundary_values}
\hat{F}^f_i
&=
\begin{cases}
\hat{\varphi}_i(0) & f = 0, \\
\hat{\varphi}_i(1) & f = 1,
\end{cases} \\
&=
\begin{cases}
\hat{P}_{i0} & f = 0, \\
\hat{P}_{iN} & f = 1,
\end{cases}
\end{align}
where $i\in\{0,1,\ldots,N\}$.
% where $l,l'\in\{0,1,\ldots,N\}$
% and $n,n'\in\{0,1,\ldots,(N+1)^d-1\}$.

We define the reference stiffness operator
$\laMat{\hat{K}}\in\mathbb{R}^{(N+1)^2}$ according to:
\begin{align}
\label{eq:ader_impl:operators:stiffness_operator_1d}
\laMatel{\hat{K}}_{ij} &=
\langle
\partial_{\hat{x}}\,\hat{\varphi}_{i},\,\hat{\varphi}_{j}\rangle_{L^2([0,1])}
\notag\\
&
\overset{\eqref{eq:ader_impl:basis:sampling_grid_1d}}
=
w_{j}\,
\partial_{\hat{x}}\,\hat{\varphi}_{i}(\hat{x}_{j}),\qquad
i,j\in\{0,1,\ldots,N\}.
\end{align}

Let us further introduce time derivative operators on the space-time reference
cell and the space-time grid cells $\cell\in\tria$:
\begin{align}
\label{eq:ader_impl:operators:stiffness_operator_time_ref}
\left\langle
\partial_{\hat{t}}\hat{\theta}_{l;n},\,
\hat{\theta}_{l';n'}
\right\rangle
_{L^2(\refCell\times[0,1])}
&=
\langle
\partial_{\hat{x}}\,\hat{\varphi}_{l},\,\hat{\varphi}_{l'}\rangle_{L^2([0,1])}
\,
\langle \hat{\phi}_{n},\,\hat{\phi}_{n'}\rangle_{L^2(\refCell)}\notag
\\
&\overset{\textup{(I)}}{=}
w_{n}\,\hat{K}_{ll'}\,\delta_{n n'},
\\
%%%
\label{eq:ader_impl:operators:stiffness_operator_time_grid}
\left\langle
\partial_{\hat{t}}\theta^{\cell;v}_{l;n},\,
\theta^{\cell;v}_{l';n'}
\right\rangle
_{L^2(\cell\times[t^\cell,t^\cell+\Delta t])}
&=
\detJ\,
\left\langle
\partial_{\hat{t}}\hat{\theta}_{l;n},\,
\hat{\theta}_{l';n'}
\right\rangle
_{L^2(\refCell\times[0,1])}
\notag
\\
&=
\detJ\,
w_{n}\,
\hat{K}_{ll'}\,
\delta_{\cell\cell'}\,\delta_{vv'}\,
\delta_{n n'},
\end{align}
where we have used the operator
\eqref{eq:ader_impl:operators:stiffness_operator_1d} and
the discrete orthogonality property of the basis functions
\eqref{eq:ader_impl:basis:discrete_ortho_ref_3d} in step I.

Let $\xi\in\{1,2,\ldots,d\}$.
We also introduce spatial derivative operators on the spatial and
space-time reference cells:
\begin{align}
\label{eq:ader_impl:operators:stiffness_operator_3d}
\left\langle
\hat{\phi}_{n},\,
\partial_{\hat{x}_\xi}\hat{\phi}_{n'}
\right\rangle
_{L^2(\refCell)}
&=
\langle
\hat{\phi}_{n},\,\partial_{\hat{x}_{\xi}}\hat{\phi}_{n'}
\rangle_{L^2(\refCell)},\notag
%%%%%%%%%%%%
\\
&=
\langle
\hat{\varphi}_{n_{\xi}},\,\partial_{\hat{x}}\hat{\varphi}_{n'_{\xi}}
\rangle_{L^2[0,1])}\notag
\,\prod_{\zeta=1,\zeta\neq\xi}^d
\langle \hat{\varphi}_{n_\zeta},\,\hat{\varphi}_{n'_\zeta}\rangle_{L^2([0,1])},
%w_l\,\partial_{\hat{x}}\varphi_{l}({\hat{t}}_{l'})\,\delta_{n n'}\,w_{n},
\notag
\\
&=
\hat{K}_{n_\xi'n_\xi }\,
\prod_{\zeta=1,\zeta\neq\xi}^d
w_{n_\zeta}\,
\delta_{n_\zeta n'_\zeta},
\\
%%%%%%%%%%%%%%%%%%%%
\label{eq:ader_impl:operators:stiffness_operator_4d}
\left\langle
\hat{\theta}_{l;n},\,
\partial_{\hat{x}_\xi}\hat{\theta}_{l';n'}
\right\rangle
_{L^2(\refCell\times[0,1])}
&=
\langle
\hat{\varphi}_{l},\,\hat{\varphi}_{l'}
\rangle_{L^2([0,1])}
\,
\langle
\hat{\phi}_{n},\,\partial_{\hat{x}_{\xi}}\hat{\phi}_{n'}
\rangle_{L^2(\refCell)},\notag
%%%%%%%%%%%%
\\
&=
w_{l}\,
\hat{K}_{n_\xi'n_\xi }\,
\delta_{ll'}\,
\prod_{\zeta=1,\zeta\neq\xi}^d
w_{n_\zeta}\,
\delta_{n_\zeta n'_\zeta}.
\end{align}

\section{The ADER-DG degrees of freedom}
We express the quantities
that are involved in the
ADER-DG scheme in terms of
the space-time and spatial basis
functions that we have introduced in
the previous section.
\begin{itemize}
  \item Space-time predictor
  $q^\cell\in\polyspace{Q}{N}{\cell\times[t^\cell,t^\cell+\Delta
  t]}^{N_\textup{var}}$:
  \begin{align*}
  q^{\cell}
  &=
  \sum_{v=0}^{N_\textup{var}}
  q^{\cell;v}
  =
  \sum_{v=0}^{N_\textup{var}}
  \sum_{l=0}^{N}
  \sum_{n=0}^{(N+1)^{d}-1}
  \widetilde{q}^{\cell;v}_{l;n}\,
  \theta^{\cell;v}_{l;n}.
  \end{align*}
  We store the space-time predictor coefficients in the vector
  $\laVec{\widetilde{q}}^\cell\in\matbb{R}^{N_\textup{var}(N+1)^{d+1}}$.
  \item Space-time volume flux
  $\stvec{F}^{\cell}\in\polyspace{Q}{N}{\cell\times[t^\cell,t^\cell+\Delta
  t]}^{N_\textup{var}\times d}$:
  \begin{align*}
  \stvec{F}^\cell
  =
  \sum_{v=0}^{N_\textup{var}}
  \stvec{F}^{K;v}
  &=
  \sum_{v=0}^{N_\textup{var}}
  \sum_{l=0}^{N}
  \sum_{n=0}^{(N+1)^d-1}
  \stvec{F}^{K;v}_{l;n}\,{\theta}^{K;v}_{l;n} \\
  &=
  \sum_{v=0}^{N_\textup{var}}
  \sum_{l=0}^{N}
  \sum_{n=0}^{(N+1)^d-1}
  (\widetilde{F}^{K;v}_{l;n,1},\ldots,\widetilde{F}^{K;v}_{l;n,d})\,{\theta}^{K;v}_{l;n}.
  \end{align*}
%   We store the space-time volume flux coefficients in the matrix\; 
%   $\laVec{\widetilde{F}}^{\cell}\in\matbb{R}^{N_\textup{var}\,(N+1)^{d+1}\times
%   d}$.
  \item Predictor $q_h^{\cell}\in\polyspace{Q}{N}{\cell}^{N_\textup{var}}$:
  \begin{align*}
  q_h^{\cell}
  &=
  \sum_{v=0}^{N_\textup{var}}
  q_h^{\cell;v}
  =
  \sum_{v=0}^{N_\textup{var}}
  \sum_{n=0}^{(N+1)^{d}-1}
  q^{\cell;v}_{n}\,
  \phi^{\cell;v}_{n}
  \end{align*}	  
  We store the predictor coefficients in the vector
  $\laVec{q}^\cell\in\matbb{R}^{N_\textup{var}(N+1)^{d}}$.
  \item Solution $u_h^{\cell}\in\polyspace{Q}{N}{\cell}^{N_\textup{var}}$:
  \begin{align*}
  u_h^{\cell}
  &=
  \sum_{v=0}^{N_\textup{var}}
  u_h^{\cell;v} 
  =
  \sum_{v=0}^{N_\textup{var}}
  \sum_{n=0}^{(N+1)^{d}-1}
  u^{\cell;v}_{n}\,
  \phi^{\cell;v}_{n}
  \end{align*}
  We store the solution coefficients in the vector
  $\laVec{u}^\cell\in\matbb{R}^{N_\textup{var}(N+1)^{d}}$.
  \item Volume flux
  $\vec{F}_h^{\cell}\in\polyspace{Q}{N}{\cell}^{N_\textup{var}\times d}$:
  \begin{align*}
  \vec{F}_h^\cell
  =
  \sum_{v=0}^{N_\textup{var}}
  \vec{F}^{K;v}
  &=
  \sum_{v=0}^{N_\textup{var}}
  \sum_{n=0}^{(N+1)^d-1}
  \vec{F}^{K;v}_{n}\,{\phi}^{K;v}_{n} \\
  &=
  \sum_{v=0}^{N_\textup{var}}
  \sum_{n=0}^{(N+1)^d-1}
  (F^{K;v}_{n,1},\ldots,F^{K;v}_{n,d})\,{\phi}^{K;v}_{n}.
  \end{align*}
%   We store the volume flux coefficients in the matrix
%   $\laMat{F}^\cell\in\matbb{R}^{N_\textup{var}\,(N+1)^{d}\times d}$.
\end{itemize}
%%%%%%%%%%%%%%%%%%%%%%%%%%%%%%%%%%%%%%%%%%%%%%%%%%%%%%%%%%%%%%%%%%%%%%%%%%%%%%%
\section{Space-time predictor computation}
%%%%%%%%%%%%%%%%%%%%%%%%%%%%%%%%%%%%%%%%%%%%%%%%%%%%%%%%%%%%%%%%%%%%%%%%%%%%%%%
The space-time predictor computation requires us to
perform $N_\textup{iter}$ Picard iterations in order
to obtain the space-time predictor
coefficients
$\widetilde{q}^{K;v}_{l;n}$,
$l\in\{0,1,\ldots,N\}$,
$n\in\{0,1,\ldots,(N+1)^{d}-1\}$:
\begin{align}
\sum_{l'=0}^{N}
\sum_{n'=0}^{(N+1)^{d}-1}
&\left(\left[\theta^{\cell;v}_{l;n},\,\theta^{\cell;v}_{l';n'}\right]^{1}_{L^2(\cell)}
-
\left\langle
\partial_{\hat{t}}\theta^{\cell;v}_{l;n},\,
\theta^{\cell;v}_{l';n'}
\right\rangle
_{L^2(\cell\times[t^\cell,t^\cell+\Delta t])}\right)
\,\widetilde{q}^{K;v;(r+1)}_{l';n'}\,
\notag
\\
&=\sum_{n'=0}^{(N+1)^{d}-1}
\left[\theta^{\cell;v}_{l;n},\,\phi^{\cell;v}_{n'}\right]^{0}_{L^2(\cell)}
\,
u^{\cell;v}_{n'}
\notag
\\
&-
\sum_{l'=0}^{N}
\sum_{n'=0}^{(N+1)^d-1}
\stvec{F}^{\cell;v}_{l'n'}
(\widetilde{q}^{\cell;(r)})
\,
\left\langle
\theta^{\cell;v}_{l';n'},\,
\nabla
\theta^{\cell;v}_{l;n}
\right\rangle
_{L^2(\cell\times[t^\cell,t^\cell+\Delta t])},
\end{align}
where $r\in\{0,1,\ldots,N_\textup{var}\}$ denotes the
current iteration.
In matrix notation, we obtain:
\begin{align}
\laVec{L}^{\cell\cell}\,\laVec{q}^{\cell;(r+1)}
&=
\laVec{v}^{K}(\laVec{u}^{\cell})-\laVec{w}^{K}(\widetilde{\laVec{q}}^{\cell;(r)})
\\
%%%
\Rightarrow
\laVec{q}^{\cell;(r+1)}
&=
{(\laVec{L}^{\cell\cell})}^{-1}\,
\left(
\laVec{v}^{K}(\laVec{u}^{\cell})-\laVec{w}^{K}(\widetilde{\laVec{q}}^{\cell;(r)})
\right)
,
\end{align}
where we identify the left-hand side operator
$\laMat{L}^{\cell\cell}\in\mathbb{R}^{N_\textup{var}\,(N+1)^{d+1}\times
N_\textup{var}\,(N+1)^{d+1}}$, and the right-hand side vectors
$\laVec{v}^{\cell}(\laVec{u}^{\cell})\in\mathbb{R}^{N_\textup{var}\,(N+1)^{d+1}}$,
and
$\laVec{w}^{\cell;(r)}(\widetilde{\laVec{q}}^{\cell;(r)})\in\mathbb{R}^{N_\textup{var}\,(N+1)^{d+1}}.
\begin{framed}
{\large \textbf{TODO}}: If we have a source term in the PDE, do we have to
consider that also in the predictor computation? I tend to say yes.
\end{framed}
%%%
\subsection{Left-hand side operator}
Let us pick two variables $v,v'\in\{0,1,\ldots,N_\textup{var}-1\}$
and two cells $\cell,\cell'\in\tria$.
The block
$\laMat{L}^{vv'}\in\mathbb{R}^{(N+1)^{d+1}\times(N+1)^{d+1}}$ has the elements:
\begin{align}
\laMatel{L}^{\cell\cell';vv'}_{ll';nn'}
&=
\left[\theta^{\cell;v}_{l;n},\,\theta^{\cell;v}_{l';n'}\right]^{1}_{L^2(\cell)}
-
\left\langle
\partial_{\hat{t}}\theta^{\cell;v}_{l;n},\,
\theta^{\cell;v}_{l';n'}
\right\rangle
_{L^2(\cell\times[t^\cell,t^\cell+\Delta t])}
\notag
\\
&
\overset{\textup{(I)}}
{=}
\detJ\,
w_{n}\,
(\hat{F}^1_{l}\,
\hat{F}^1_{l'}\,
-\hat{K}_{ll'})\,
\delta_{\cell\cell'}\,\delta_{vv'}\,
\delta_{n n'}
\notag
\\
&{=}
\detJ\,
w_{n}\,
\hat{L}_{ll'}\,
\delta_{\cell\cell'}\,\delta_{vv'}\,
\delta_{n n'},
\end{align}
where we have introduced the reference cell operator
\begin{align}
\hat{L}_{ll'}
&=
\hat{F}^1_{l}\,
\hat{F}^1_{l'}\,
-\hat{K}_{ll'},
\end{align}
where $l,l'\in\{0,1,\ldots,N\}$, and
$n,n'\in\{0,1,\ldots,(N+1)^{d}-1\}$.
In step I of the above derivations, we
used
\eqref{eq:ader_impl:operators:tau_operator_1_grid}
and
\eqref{eq:ader_impl:operators:stiffness_operator_time_grid}.
\subsection{Constant right-hand side term}
The elements of vector
$\laVec{v}^{\cell;v}$, $v\in\{0,1,\ldots,N_\textup{var}\}$, are computed
according to:
\begin{align*}
\laVecel{v}^{\cell;v}_{l;n}
&{=}
\sum_{n'=0}^{(N+1)^{d}-1}
\left[\theta^{\cell;v}_{l;n},\,\phi^{\cell;v}_{n'}\right]^{0}_{L^2(\cell)}
\,
u^{\cell;v}_{n'}
\\
%%%
&\overset{\eqref{eq:ader_impl:operators:tau_operator_1_grid}}=
\sum_{n'=0}^{(N+1)^{d}-1}
\detJ\,
w_{n}\,
\hat{F}^0_l\,
\delta_{\cell\cell'}\,\delta_{vv'}\,
\delta_{n n'}
\,
u^{\cell;v}_{n'}\,
\\
%%%
&=
\detJ\,
w_{n}\,
\hat{F}^0_l\,
u^{\cell;v}_{n},
\end{align*}
where $l\in\{0,1,\ldots,N\}$, and
$n\in\{0,1,\ldots,(N+1)^{d}-1\}$.
%%%%%%%%%%%%%%%%%%%%%%%%%%%%%%%%%%%%%%%%%%%%%%%%%%%%%%%%%%%%%%%%%%%%%%%%%%%%%%%
\subsection{Space-time volume flux integral}
\label{sec:ader_impl:predictor:space_time_volume_flux_integral}
%%%%%%%%%%%%%%%%%%%%%%%%%%%%%%%%%%%%%%%%%%%%%%%%%%%%%%%%%%%%%%%%%%%%%%%%%%%%%%%
The elements of vector $\laVec{w}^{\cell;v}$,
$v\in\{0,1,\ldots,N_\textup{var}\}$, are computed according to:
\begin{align}
\laVecel{w}^{\cell;v}_{l;n}
&{=}
\sum_{l'=0}^{N}
\sum_{n'=0}^{(N+1)^d-1}
\stvec{F}^{K;v}_{l'n'}\,
\left\langle
\theta^{K;v}_{l';n'},\,
\nabla
\theta^{K;v}_{l;n}
\right\rangle
_{L^2(\cell\times[t^\cell,t^\cell+\Delta t])}
%\,
%\dV
%\,
%\dt
\notag
\\
&\overset{(\textup{I})}
{=}
%\int_{0}^{1}
\Delta t\,
%\int_{\refCell}
\detJ
%\left(\frac{\partial\,\vec{x}}{\partial\,\refVec{x}}\right)^{-\textup{T}}
\sum_{l'=0}^{N}
\sum_{n'=0}^{(N+1)^d-1}
\stvec{F}^{K;v}_{l'n'}\,
\jacobian^{-\textup{T}}\,
\left\langle
\hat{\theta}_{l';n'}\,
\hat{\nabla}
\hat{\theta}_{l;n}
\right\rangle
_{L^2(\refCell\times[0,1])}
%\,
%\refdV
%\,
%\refdt
\notag
\\
%%%
&\overset{(\textup{II})}
{=}
\detJ\,
\sum_{l'=0}^{N}
\sum_{n'=0}^{(N+1)^d-1}
\sum_{\xi=1}^d
\frac{\Delta t}{\Delta x_{\xi}}\,
\left\langle
\hat{\theta}_{l';n'},\,
\partial_{\hat{x}_\xi}\hat{\theta}_{l;n}
\right\rangle
_{L^2(\refCell\times[0,1])}
\;\widetilde{F}^{\cell;v}_{l';n',\xi}
\notag
\\
%%%
&\overset{(\textup{III})}
{=}
\detJ\,
\sum_{l'=0}^{N}
\sum_{n'=0}^{(N+1)^d-1}
\sum_{\xi =1}^d
\frac{\Delta t}{\Delta x_{\xi}}\,
\laMatel{\hat{K}}_{n_\xi n'_\xi}\,
\widetilde{F}^{\cell;v}_{l';n',\xi}\,
w_{l}\,
\delta_{ll'}\,
\prod_{\zeta=1,\zeta\neq\xi}^d
w_{n_\zeta}\,
\delta_{n'_\zeta n_\zeta},
\end{align}
where $l\in\{0,1,\ldots,N\}$, and
$n\in\{0,1,\ldots,(N+1)^{d}-1\}$.
In step I of the above derivations, we applied a scaling argument.
In step II, we expanded the $d$-dimensional scalar
product and wrote the integral over the space-time reference cell as a
discrete scalar product.
In step III, we used \eqref{eq:ader_impl:operators:stiffness_operator_4d}.

%%%%%%%%%%%%%%%%%%%%%%%%%%%%%%%%%%%%%%%%%%%%%%%%%%%%%%%%%%%%%%%%%%%%%%%%%%%%%%%
\section{Time averaging of the space-time predictor values}
%%%%%%%%%%%%%%%%%%%%%%%%%%%%%%%%%%%%%%%%%%%%%%%%%%%%%%%%%%%%%%%%%%%%%%%%%%%%%%%
In the current implementation, we need to compute the time average of the
space-time predictor values.
This computation must be performed before the boundary extrapolation of
the predictor values and before the volume integral that also
relies on the boundary extrapolated predictor values.

Let us express a component of the local space-time predictor in terms
of the local space-time basis
\begin{align*}
q^{\cell;v}
=
\sum_{l'=0}^{N}
\sum_{n'=0}^{(N+1)^{d}-1}
\widetilde{q}^{\cell;v}_{l';n'}\,
\theta^{\cell;v}_{l';n'}
\end{align*}

We want to compute the predictor in $[t^\cell,t^\cell+\Delta t]$
as the time average of the space-time predictor over the same interval, i.e.,

\begin{align*}
q^{\cell;v}_h
&=
\sum_{n'=0}^{(N+1)^{d}-1}
q^{\cell;v}_{n'}\,
\phi^{\cell;v}_{n'}\,
\\
&=
\frac{1}{\Delta t}
\int_{t^\cell}^{t^\cell+\Delta t}
\sum_{l'=0}^{N}
\sum_{n'=0}^{(N+1)^{d}-1}
\widetilde{q}^{\cell;v}_{l';n'}\,
\theta^{\cell;v}_{l';n'}\,
\dt
\\
&=
\sum_{n'=0}^{(N+1)^{d}-1}
\phi^{\cell;v}_{n'}\,
\frac{1}{\Delta t}
\int_{t^\cell}^{t^\cell+\Delta t}
\sum_{l'=0}^{N}\,
\widetilde{q}^{\cell;v}_{l';n'}\,
\varphi^{\cell;v}_{l'}\,
\dt
\\
&=
\sum_{n'=0}^{(N+1)^{d}-1}
\phi^{\cell;v}_{n'}\,
\frac{1}{\Delta t}
\sum_{l'=0}^{N}\,
\widetilde{q}^{\cell;v}_{l';n'}\,
\langle
1,\,
\varphi^{\cell;v}_{l'}\,
\rangle_{L^2[t^\cell,t^\cell+\Delta t]}
\\
&\overset{\eqref{eq:ader_impl:basis:sampling_grid_1d}}
{=}
\sum_{n'=0}^{(N+1)^{d}-1}
\phi^{\cell;v}_{n'}\,
\sum_{l'=0}^{N}\,
w_{l'}\,
\widetilde{q}^{\cell;v}_{l';n'}.
\end{align*}
Thus,
\begin{align}
q^{\cell;v}_{n'}
&=\sum_{l'=0}^{N}\,
w_{l'}\,
\widetilde{q}^{\cell;v}_{l';n'},\qquad n=0,1,\ldots,(N+1)^d-1.
\end{align}
A similar expression can be derived for the time average of
the space-time volume flux components:
\begin{align}
\textbf{F}^{\cell;v}_{n'}
&=\sum_{l'=0}^{N}\,
w_{l'}\,
\widetilde{\textbf{F}}^{\cell;v}_{l';n'},\qquad n=0,1,\ldots,(N+1)^d-1.
\end{align}
%%%%%%%%%%%%%%%%%%%%%%%%%%%%%%%%%%%%%%%%%%%%%%%%%%%%%%%%%%%%%%%%%%%%%%%%%%%%%%%
\section{Boundary Extrapolation}
%%%%%%%%%%%%%%%%%%%%%%%%%%%%%%%%%%%%%%%%%%%%%%%%%%%%%%%%%%%%%%%%%%%%%%%%%%%%%%%
The boundary extrapolation can be in some sense
interpreted as the ``transposed'' operation to
the surface integral discussed in the next section.

Let us define on the reference element $2\,d$ sets of support points
\begin{align*}
\{\refVec{x}_{k}\}&_{k=0,1,\ldots,(N+1)^{d-1}-1}^{\xi f}
\\
&=
\left\{\refVec{x}_{n}\colon\; \hat{x}_{n,\xi}=0\textup{ if }f=0
\textup{ or } \hat{x}_{n,\xi}=1 \textup{ if }f=1,\;n=0,1,\ldots,(N+1)^{d}-1
\right\},
\end{align*}
where $\xi=1,2,\ldots,d$, and $f=0,1$.
We construct the corresponding support points on the grid cells $\cell\in\tria$
by applying a mapping:
\begin{align*}
\{\vec{x}_{k}\}&_{k=0,1,\ldots,(N+1)^{d-1}-1}^{\xi f}
\\
&=
\mapping
{\{\refVec{x}_{k}\}_{k=0,1,\ldots,(N+1)^{d-1}-1}^{\xi f}}
\end{align*}

The boundary extrapolation requires us to sum the contributions of the basis
functions at the quadrature points on each face $\face^{\xi f}$.

Let us start with the extrapolation of the
normal flux tensor components
$
(\vec{F}^{\cell;v}\,\vec{n}^{\xi})\in\mathbb{R}^{(N+1)^{d}},
$
$v=1,2,\ldots,N_\textup{var}$,
$\xi\in\{1,2,\ldots,d\}$,
which are scalar-valued quantities.

Using the quadrature nodes
$\vec{x}^{\xi f}_k\in\{\refVec{x}_{k}\}_{k=0,1,\ldots,(N+1)^{d-1}-1}^{\xi f}$
,$k=0,1,\ldots,(N+1)^{d-1}-1$,
located on face $e^{\xi f}$,
$\xi=1,2,\ldots,d$, $f=0,1$,
we obtain for the components of the normal flux vector
$\laVec{g}^{K;\xi f}\in\mathbb{R}^{N_\textup{var}\,(N+1)^{d-1}}$:
\begin{align*}
\laVecel{g}^{K;\xi f;v}_k
&=
\sum_{n'=0}^{(N+1)^{d}-1}
(\vec{F}^{\cell;v}\,\vec{n}^{\xi})_{n'}\,
\phi^{\cell;v}_{n'}(\vec{x}^{\xi f}_k)
\\
&
\overset{\textup{(I)}}
{=}
\sum_{n'=0}^{(N+1)^{d}-1}
(\vec{F}^{\cell;v}\,\vec{n}^{\xi})_{n'}\,
\hat{\phi}_{n'}(\refVec{x}^{\xi f}_k)
\\
&
\overset{\textup{(II)}}
{=}
\sum_{n'=0}^{(N+1)^{d}-1}
(\vec{F}^{\cell;v}\,\vec{n}^{\xi})_{n'}\,
\hat{\varphi}_{n'_\zeta}(\hat{x}^{\xi f}_{k,\xi})
\prod_{\zeta=1,\zeta\neq\xi}^{d}
\hat{\varphi}_{n'_\zeta}(\hat{x}^{\xi f}_{k,\zeta})
\\
&
\overset{\textup{(III)}}
{=}
\sum_{n'=0}^{(N+1)^{d}-1}
(\vec{F}^{\cell;v}\,\vec{n}^{\xi})_{n'}\,
\hat{\varphi}_{n'_\xi}(\hat{x}^{\xi f}_{k,\xi})
\prod_{\zeta=1,\zeta\neq\xi}^{d}
\delta_{n'_\zeta k_\zeta}
\\
&
\overset{\phantom{\textup{(IV)}}}
{=}
\sum_{n'=0}^{(N+1)^{d}-1}
(\vec{F}^{\cell;v}\,\vec{n}^{\xi})_{n'}\,
\hat{F}^f_{n'_\xi}
\prod_{\zeta=1,\zeta\neq\xi}^{d}
\delta_{n'_\zeta k_\zeta},
\end{align*}
where $k\in\{1,2,\ldots,(N+1)^{d-1}-1\}$, and
where the vectors $\hat{F}^f \in \mathbb{R}^{N+1}$,
$f=0,1$, are defined by \eqref{eq:ader_impl:operators:boundary_values}.
In step I of the above calculations, we made use of the definition of the
multivariate basis functions; cf.~\eqref{eq:ader_impl:basis:basis_definition_grid_3d}.
In step II, we split the multivariate reference basis functions
into the univariate ones; cf.~\eqref{eq:ader_impl:basis:basis_definition_ref_3d}.
In step III, we made use of the Lagrange basis property
\eqref{eq:ader_impl:basis:lagrange_ref_1d}.

We obtain for the elements of the extrapolated
predictor vectors $\laVec{e}^{K;\xi
f}\in\mathbb{R}^{N_\textup{var}\,(N+1)^{d-1}}$ $\xi=1,2,\ldots,d$, $f=0,1$
:
\begin{align*}
\laVecel{e}^{K;\xi f;v}_{k}
&=
\sum_{n'=0}^{(N+1)^{d}-1}
q^{\cell;v}_{n'}\,
\phi^{\cell;v}_{n'}(\vec{x}^{\xi f}_k)
\\
&
\overset{\phantom{\textup{(IV)}}}
{=}
\sum_{n'=0}^{(N+1)^{d}-1}
q^{\cell;v}_{n'}\,
\hat{F}^f_{n'_\xi}
\prod_{\zeta=1,\zeta\neq\xi}^{d}
\delta_{n'_\zeta k_\zeta},
\end{align*}
where $k\in\{1,2,\ldots,(N+1)^{d-1}-1\}$.

Remarks:
\begin{itemize}
  \item Note that the predictor values used for the boundary extrapolation might
  be replaced by time-integrated space-time predictor values if
  we want to employ anarchic time stepping.
\end{itemize}

%%%%%%%%%%%%%%%%%%%%%%%%%%%%%%%%%%%%%%%%%%%%%%%%%%%%%%%%%%%%%%%%%%%%%%%%%%%%%%%
\section{Correction}
%%%%%%%%%%%%%%%%%%%%%%%%%%%%%%%%%%%%%%%%%%%%%%%%%%%%%%%%%%%%%%%%%%%%%%%%%%%%%%%
The correction phase requires us to solve the following system
of equations
to obtain the solutions
coefficients
$u^{\cell;v}_{n}$,
$n\in\{0,1,\ldots,(N+1)^{d}-1\}$:
\begin{align}
\sum_{n'=0}^{(N+1)^d-1}
&\left\langle 
\phi^{\cell;v}_{n},\,\phi^{\cell;v}_{n'}
\right\rangle
_{L^2(\cell)}
\,
(u^{\cell;v}_{n}-u^{\cell;v(\textup{old})}_{n})
\notag
\\
%%%%%%%
&+
\int_{t^\cell}^{t^\cell+\Delta t}
\int_\cellBnd
\phi^{\cell;v}_{n}\,\vec{G}(q^{\cell+}_h,q^{\cell-}_h\,\vec{n})\,
\dV\,\dt
\notag
\\
%%%%%
&-
\int_{t^\cell}^{t^\cell+\Delta t}
\int_\cell
\vec{F}(q^{\cell}_h)\,
\nabla \phi^{\cell;v}_{n}\,
\dV\,\dt  = 0,
\end{align}
The correction step thus requires us to solve the following
equation for $\laVec{u}^{\cell}$
\begin{align*}
\laMat{M}^{\cell\cell}\,
(\laVec{u}^{\cell} - \laVec{u}^{\cell;(\textup{old})})
&= \Delta t\,\left (
\laVec{\Delta u}^{\cell}
%+
%\laVec{S}^{\cell}
\right)
\end{align*}
where the solution update vector is constructed according to:
\begin{align*}
\laVec{\Delta u}^{\cell}
&=
\laVec{a}^{\cell}
-
\laVec{b}^{\cell}
\end{align*}
All the operators and vectors introduced above will be discussed
in detail in the next sections. 
\begin{framed}
{\large \textbf{TODO}}: If we have a source term in the PDE, do we have to
consider that also in the predictor computation? I tend to say yes.
\end{framed}
%%%%%%%%%%%%%%%%%%%%%%%%%%%%%%%%%%%%%%%%%%%%%%%%%%%%%%%%%%%%%%%%%%%%%%%%%%%%%%%
\subsection{Mass operator}
%%%%%%%%%%%%%%%%%%%%%%%%%%%%%%%%%%%%%%%%%%%%%%%%%%%%%%%%%%%%%%%%%%%%%%%%%%%%%%%
The mass operator consists  of ${(\Ncell\cdot N_\textup{var})^2}$ blocks with
size $(N+1)^d$, where $N_\cell$ denotes the number of elements and
$d$ denotes the space dimension.

Let us pick two variables $v,v'\in\{0,1,\ldots,N_\textup{var}-1\}$
and two grid cells $\cell,\cell'\in\tria$.
The block $\laMat{M}^{\cell\cell';vv'}\in\mathbb{R}^{(N+1)^{d}\times(N+1)^{d}}$
has the elements:
\begin{align}
\laMatel{M}^{\cell\cell';vv'}_{nn'}
\overset{\phantom{\textup{\eqref{eq:ader_impl:basis:orthogonal_variables}}}}{=}&
\langle {\phi}^{\cell';v'}_{n'},{\phi}^{\cell;v}_{n}\rangle
_{L^2 (\cell) }\,
\notag
\\
%%%%%%%%%%%%%%
\label{eq:ader_impl:correction:mass_operator}
\overset{\textup{\eqref{eq:ader_impl:basis:orthogonal_variables}}}{=}&
\delta_{\cell\cell'}\,\delta_{vv'}\,\delta_{nn'}\,
J_\cell\,w_{n}.
\end{align}
The mass matrix is thus completely diagonal.

%%%%%%%%%%%%%%%%%%%%%%%%%%%%%%%%%%%%%%%%%%%%%%%%%%%%%%%%%%%%%%%%%%%%%%%%%%%%%%%
\subsection{Volume integral}
%%%%%%%%%%%%%%%%%%%%%%%%%%%%%%%%%%%%%%%%%%%%%%%%%%%%%%%%%%%%%%%%%%%%%%%%%%%%%%%
The evaluation of the volume integral is very similar 
to the evaluation of the space-time volume flux integral
(Section \ref{sec:ader_impl:predictor:space_time_volume_flux_integral}).

The compact support of the DG basis functions renders the computation of the
volume flux vector $\laVec{a}\in\mathbb{R}^{N_\cell
N_\textup{var}\,(N+1)^d}$ a cell-local operation.

The cell-local volume flux vector $\laVec{a}^{\cell;v}$ is computed
according to:
\begin{align*}
\laVecel{a}^{\cell;v}_{n}
&=
% \int_\cell
\sum_{n'=0}^{(N+1)^d-1}
\vec{F}^{K;v}_{n'}\,
\left\langle
{\phi}^{K;v}_{n'}\,
\nabla {\phi}^{K;v}_{n}\,
\right\rangle
_{L^2(\cell)}
% \dV
\\
&\overset{(\textup{I})}
{=}
J_\cell
%\int_{\refCell}
\sum_{n'=0}^{(N+1)^d-1}
\vec{F}^{K;v}_{n'}\,
\jacobian^{-\textup{T}}\,
%\left(\frac{\partial\,\vec{x}}{\partial\,\refVec{x}}\right)^{-\textup{T}}
\left\langle
\hat{\phi}_{n'},\,
\hat{\nabla}
\hat{\phi}_{n}\,
\right\rangle
_{L^2(\refCell)}
\,
%\dV
\\
%%%
&\overset{(\textup{II})}
{=}
J_\cell
\sum_{\xi=1}^d
\frac{1}{\Delta x_\xi}
\sum_{n'=0}^{(N+1)^d-1}
\langle
\hat{\phi}_{n'},\,
\partial_{\hat{x}_{\xi}}\,\hat{\phi}_{n}
\rangle_{L^2(\refCell)}
\;F^{\cell;v}_{n',\xi}
\\
%%%
&\overset{(\textup{III})}
{=}
J_\cell
\sum_{\xi =1}^d
\frac{1}{\Delta x_\xi}
\sum_{n'=0}^{(N+1)^d-1}
\laMatel{\hat{K}}_{n_\xi n'_\xi}\,
F^{\cell;v}_{n',\xi}\,
\,\prod^{d}_{\zeta=1,\zeta\neq\xi}\,w_{n'_\zeta}\,\delta_{n'_\zeta\,n_\zeta},
\end{align*}
In step I of the above derivations, we used a scaling argument.
In step II, we expanded the $d$-dimensional scalar product and
wrote the integral over the reference element as a discrete scalar product.
In step III, we used \eqref{eq:ader_impl:operators:stiffness_operator_3d}.

%%%%%%%%%%%%%%%%%%%%%%%%%%%%%%%%%%%%%%%%%%%%%%%%%%%%%%%%%%%%%%%%%%%%%%%%%%%%%%%
\subsection{Surface integral}
%%%%%%%%%%%%%%%%%%%%%%%%%%%%%%%%%%%%%%%%%%%%%%%%%%%%%%%%%%%%%%%%%%%%%%%%%%%%%%%
Let us introduce positively signed normal vectors $\vec{n}^{\xi}\in\mathbb{R}^d$
that have their only non-zero at position $\xi$.
Let us further introduce outward directed normal vectors $\vec{n}^{\xi
f}\in\mathbb{R}^d$ that also have their only non-zero at position $\xi$. Their
sign is however negative if $f=0$ and positive if $f=1$.

The face fluctuations must be unique for two cells that share an interior
face. Here, they are defined with respect to one of the directions
$\vec{n}^{\xi}\in\mathbb{R}^d$.
From this definition of the fluctuations, it follows that
the face fluctuations with respect to the outward directed
normal vectors can be computed according to:
\begin{align}
(\vec{G}^{\cell;v}\,\vec{n}^{\xi f})
&=
\begin{cases}
-(\vec{G}^{\cell;v}\,\vec{n}^{\xi}) & f=0, \\
+(\vec{G}^{\cell;v}\,\vec{n}^{\xi}) & f=1.
\end{cases},
\end{align}
where $(\vec{G}^{\cell;v}\,\vec{n}^{\xi})\in\mathbb{R}^{(N+1)^{d-1}}$
denote the face fluctuations, and $\vec{n}^{\xi f}$ denote the
outward directed normal vectors, $\xi\in\{1,2,\ldots,d\}$, $f\in\{0,1\}$.

Evaluating the values of the grid cell basis functions on a face of the mesh,
leads to the evaluation of the reference basis functions at
either $\hat{x}_\xi=0$ or $\hat{x}_\xi=1$, $\xi\in\{1,2,\ldots,d\}$.

The result of the projection of the boundary fluctuation values on the
degrees of freedom is thus computed according to:
\begin{align}
\hat{F}^f\otimes(\vec{G}^{\cell;v}\,\vec{n}^{\xi
f})\in\mathbb{R}^{(N+1)^d},
\end{align}
where $\hat{F}^f\in\mathbb{R}^{N+1}$ is defined
by \eqref{eq:ader_impl:operators:boundary_values}.

Let us now procede to derive the cell-local surface integral contributions to
the fluctuations vector $\laVec{b}\in \mathbb{R}^{N_\cell
N_\textup{var}\,(N+1)^d}$.
The vector has the entries
\begin{align*}
\laVecel{b}^{\cell;v}_n &=
\sum_{\xi=1}^{d}
\sum_{f=0}^{1}
\int_{\face^{\xi f}} {\phi}^{K;v}_{n}\,
\hat{F}^f\otimes(\vec{G}^{\cell;v}\,\vec{n}^{\xi
f})
\,\ds
\\
&=
\sum_{\xi=1}^{d}
\sum_{f=0}^{1}
\int_{\face^{\xi f}} {\phi}^{K;v}_{n}\,
\sum_{n'=0}^{(N+1)^{d}-1}
(\hat{F}^f\otimes(\vec{G}^{\cell;v}\,\vec{n}^{\xi
f}))_{n'}
\,
{\phi}^{K;v}_{n'}\,\ds
\\
&=
\sum_{\xi=1}^{d}
\sum_{f=0}^{1}
\sum_{i=0}^{(N+1)^{d-1}-1}
|e^{\xi f}|\,
\prod_{\zeta=1,\zeta\neq\xi}^d
w_{i_\zeta}\,
\sum_{n'=0}^{(N+1)^{d}-1}
\hat{\phi}_{n}(\refVec{x}_i)\,
(\hat{F}^f\otimes(\vec{G}^{\cell;v}\,\vec{n}^{\xi
f}))_{n'}
\,
\hat{\phi}_{n'}(\refVec{x}_i)
\\
&=
\sum_{\xi=1}^{d}
\sum_{f=0}^{1}
\sum_{i=0}^{(N+1)^{d-1}-1}
\prod_{\zeta=1,\zeta\neq\xi}^d
\Delta x_{\zeta}\,
w_{i_\zeta}\,
\sum_{n'=0}^{(N+1)^{d}-1}
\delta_{n i}\,
(\hat{F}^f\otimes(\vec{G}^{\cell;v}\,\vec{n}^{\xi
f}))_{n'}
\,
\delta_{n' i}
\\
&=
\sum_{\xi=1}^{d}
\sum_{f=0}^{1}
\prod_{\zeta=1,\zeta\neq\xi}^d
\Delta x_{\zeta}\,
w_{n_\zeta}\,
(\hat{F}^f\otimes(\vec{G}^{\cell;v}\,\vec{n}^{\xi
f}))_{n},
\end{align*}
for $n=1,2,\ldots,(N+1)^{d}-1$.
%%%%%%%%%%%%%%%%%%%%%%%%%%%%%%%%%%%%%%%%%%%%%%%%%%%%%%%%%%%%%%%%%%%%%%%%%%%%%%%
\subsection{Source terms}
%%%%%%%%%%%%%%%%%%%%%%%%%%%%%%%%%%%%%%%%%%%%%%%%%%%%%%%%%%%%%%%%%%%%%%%%%%%%%%%
Assume that the source term $\textbf{S}\in\mathbb{R}^{N_\textup{var}}$
is sufficiently regular such that it can be approximated by the DG basis in each
cell $\cell\in\tria$.
Let $S^v$, $v=0,1,\ldots,N_\textup{var}-1$ denote a component of the source term
and let us further define $S^{\cell;v} = S^v|_\cell$
and $S^{\cell;v}_n = S^{\cell;v}(\vec{x}_n)$.

We can express the source term components according to:
\begin{align*}
S^{\cell;v}
&{\approx}
\sum_{n=0}^{(N+1)^d-1}
\frac
{
\left\langle S^{\cell;v},\,{\phi}^{\cell;v}_{n}
\right\rangle_{L^2(\cell)}
}
{
\left\langle{\phi}^{\cell;v}_{n},{\phi}^{\cell;v}_{n}
\right\rangle_{L^2(\cell)}
}
{\phi}^{\cell;v}_{n}
\\
%%%%%%%
&\overset{\textup{(I)}}{=}
\sum_{n=0}^{(N+1)^d-1}
\frac
{
S^{\cell;v}_{n}\,w_{n}
}
{
w_{n}
}
{\phi}^{\cell;v}_{n}
\\
%%%%%%
&{=}
\sum_{n=0}^{(N+1)^d-1}
S^{\cell;v}_{n}\,\phi^{\cell;v}_{n}.
\end{align*}
where we have used the sampling property
\eqref{eq:ader_impl:basis:sampling_grid_3d}, and
the discrete orthogonality property
\eqref{eq:ader_impl:basis:discrete_ortho_grid_3d}
of the basis polynomials in step I.

We thus obtain for the elements of the source term contribution vector
$\laVec{s}\in\mathbb{R}^{\Ncell\,N_\textup{var}\,(N+1)^d}$:
\begin{align*}
\laVecel{s}^{K;v}_n =
\int_\cell S^{\cell;v}\,{\phi}^{\cell;v}_{n}\,
\dV
\overset{\phantom{\textup{\eqref{eq:ader_impl:basis:orthogonal_variables}}}}{=}&
\langle\,S^{\cell;v},\,{\phi}^{\cell;v}_{n}\,
\rangle_{L^2(\cell)}\\
\overset{\phantom{\textup{\eqref{eq:ader_impl:basis:orthogonal_variables}}}}{=}&
\sum_{n'=1}^{(N+1)^d-1}
S^{\cell;v}_{n'}\,
\langle
\phi^{\cell;v}_{n'},\,
\phi^{\cell;v}_{n}\,
\rangle_{L^2(\cell)}
\\
\overset{\textup{\eqref{eq:ader_impl:basis:discrete_ortho_grid_3d}}}{=}&
\sum_{n'=1}^{(N+1)^d-1}
\delta_{n n'}\,
\detJ\,
w_{n'}\,
S^{\cell;v}_{n'}\,
\\
\overset{\phantom{\textup{\eqref{eq:ader_impl:basis:orthogonal_variables}}}}{=}&
w_n\,
J_\cell\,
S^{\cell;v}_{n},
\end{align*}
where $\cell\in\tria$, $v=0,1\ldots,N_\textup{var}$, and
$n=0,1,\ldots,(N+1)^{d}-1$.
%%%%%%%%%%%%%%%%%%%%%%%%%%%%%%%%%%%%%%%%%%%%%%%%%%%%%%%%%%%%%%%%%%%%%%%%%%%%%%%
\section{Solution update}
%%%%%%%%%%%%%%%%%%%%%%%%%%%%%%%%%%%%%%%%%%%%%%%%%%%%%%%%%%%%%%%%%%%%%%%%%%%%%%%
Let us define the solution update vector as
\begin{align*}
\laVec{\Delta u}^{\cell}
&=
\laVec{a}^{\cell}
-
\laVec{b}^{\cell}
\end{align*}
The correction step requires us to solve the following
equation for $\laVec{u}^{\cell}$
\begin{align*}
\laMat{M}^{\cell\cell}\,
(\laVec{u}^{\cell} - \laVec{u}^{\cell;(\textup{old})})
&= \Delta t\,\left (
\laVec{\Delta u}^{\cell}
+
\laVec{S}^{\cell}
\right)
\\
\intertext{Fixing the variable $v'\in{0,1,\ldots,N_\textup{var}}$
and the degree of freedom $n'\in{0,1,\ldots,(N+1)^d-1}$,
and expanding the matrix-vector product leads to (cf.
\eqref{eq:ader_impl:correction:mass_operator}):
}
\sum_{v=0}^{N_\textup{var}}
\,
\sum_{n=0}^{(N+1)^d-1}
\delta_{vv'}\,\delta_{nn'}\,
J_\cell\,w_{n}\,
(\laVecel{u}^{\cell;v'}_{n'} -
\laVecel{u}_{n'}^{\cell;v';(\textup{old})}) &= \Delta t\,\left (
\laVecel{\Delta u}^{\cell;v'}_{n'}
+
\laVecel{S}^{\cell;v'}_{n'}
\right)
\\
\Rightarrow J_\cell\,w_{n}\,
(\laVecel{u}^{\cell;v}_{n} - \laVecel{u}_{n}^{\cell;v;(\textup{old})})
&= \Delta t\,\left (
\laVecel{\Delta u}^{\cell;v}_{n}
+
\laVecel{S}^{\cell;v}_{n}
\right)
\end{align*}
From rearranging the equation, we finally obtain
\begin{align*}
\laVecel{u}^{\cell;v}_{n}
&=
\laVecel{u}_{n}^{\cell;v;(\textup{old})}
+
\frac{\Delta t}{w_{n}}\,\left (
\frac{1}{J_\cell}
\laVecel{\Delta u}^{\cell;v}_{n}
+
\frac{1}{J_\cell}
\laVecel{S}^{\cell;v}_{n}
\right),
\end{align*}
where $v\in{0,1,\ldots,N_\textup{var}}$
and $n\in{0,1,\ldots,(N+1)^d-1}$.
Notice the division by $\detJ$ on the right-hand side.

%%%%%%%%%%%%%%%%%%%%%%%%%%%%%%%%%%%%%%%%%%%%%%%%%%%%%%%%%%%%%%%%%%%%%%%%%%%%%%%
\section{Projecting the ADER-DG degrees of freedom on a regular grid (``Sub
output matrix'')}
%%%%%%%%%%%%%%%%%%%%%%%%%%%%%%%%%%%%%%%%%%%%%%%%%%%%%%%%%%%%%%%%%%%%%%%%%%%%%%%
The DG degrees of freedom can be projected on a regular partition
of a cell $\cell\in\tria$
using a $d$-dimensional tensor product of the 1-$d$
regular grid projector defined
by \eqref{eq:ader_impl:operators:regular_grid_projector_1d}.
 
Let $u_h^{(\textup{reg})}\in\polyspace{Q}{N}{{(N+1)^d}}$ denote the
solution $u_h$ projected on a regular partition of $\cell$.

We compute the coefficients of $u_h^{(\textup{reg})}$
according to:
\begin{align}
u_{h;n}^{(\textup{reg})}
&\overset{\eqref{eq:ader_impl:basis:basis_definition_grid_3d}}
{=}
\sum_{n'=0}^{(N+1)^d-1}
u_{h;n'}
\prod_{\xi=1}^{d}
\hat{P}_{n'_\xi,n_{\xi}}
,
\qquad n=0,1,\ldots,(N+1)^d-1,
\end{align}

\section{Remarks on the current ExaHyPE implementation}
\begin{enumerate}
\item If we store the 1-$d$ regular grid projector $\hat{P}$, we
do not need to store the operators $\hat{F}^f$, $f=0,1$
(\texttt{FLCoeff},\texttt{FRCoeff},\texttt{FCoeff}).
We further need not to store the large sub output operators
(\texttt{subOutputMatrix})
which are $d$-dimensional tensor products of $\hat{P}$.
\item There is no need to allocate extra memory for the time 
boundary values, i.e., $\hat{F}^0$ (\texttt{F0})
since these values are already stored
in \texttt{FLCoeff}, and \texttt{FCoeff}. 
\item \texttt{FLCoeff} and \texttt{FRCoeff}  
contain the same information as \texttt{FCoeff}. 
\end{enumerate}

Conclusion:
\begin{itemize}
\item
The operators
\texttt{FLCoeff}$\in\mathbb{R}^{N+1}$,\texttt{FRCoeff}
$\in\mathbb{R}^{N+1}$,\texttt{FCoeff}$\in\mathbb{R}^{2\,(N+1)}$,\\
and \texttt{subOutputMatrix}$\in\mathbb{R}^{(N+1)^d\times(N+1)^d\times\ldots}$
can all be realised by storing a 1-$d$ regular grid projector
$\hat{P}\in\mathbb{R}^{(N+1)\times(N+1)}$.
\end{itemize}

\end{document}
